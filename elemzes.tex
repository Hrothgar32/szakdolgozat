%!TEX root = dolgozat.tex
%%%%%%%%%%%%%%%%%%%%%%%%%%%%%%%%%%%%%%%%%%%%%%%%%%%%%%%%%%%%%%%%%%%%%%%
\chapter{Eredmények bemutatása és értékelése}\label{ch:elemzes}


\section{Az utazóügynök feladata}

A standard utazóügynök feladat a következõ:

Adott egy súlyozott gráf G = (V,E) a cij súly az i és j csomópontokat
összekötõ élre vontakozik, és a cij érteke egy pozitiv szám. Találd meg azt a
körutat, amelynek minimális a költsége.

\section{Az utazóügynök feladatára vonatkozó heurisztikák}

Az utazóügynök feladata egy np-teljes feladat. A megoldás megtalálására túl
sok idõ szükséges, ezért heurisztikákat használunk.

\subsection{Beszúrási herusztika}

A beszúrási heurisztikát akkor használjuk, amikor egy új 
csomópontot akarunk beszúrni a körútba. úgy szúrunk 
be egy új csomópontot , hogy a körút hossza minimális 
legyen. A csomópontot minden pozícióba megpróbáljuk 
beszúrni, és mindig kiszámoljuk a költséget. Az új 
pozícioja a csomópontnak a körútban az a pozício lesz, 
ahol a költség minimális.

\subsection{Körútjavító heurisztika}

A körút javito heurisztikakat arra használjuk, hogy meglévõ 
megoldásokat javitsunk. A legismertebb heurisztikak a 2-opt és a 
3-opt heurisztikak.

\subsubsection{A 2-opt herurisztika}

A 2 opt heurisztika megprobal talalni 2 elet, amelyet el lehet tavolitani, 
és 2 elet, amelyet be lehet szúri, ugy, hogy egy körútat 
kapjunk, amelynek a költsége kisebb mint az eredetié. Euklideszi 
távolságoknál a 2-opt csere, kiegyenesíti a 
körútat, amely saját magát keresztezi.

A 2-opt algoritmus tulajdonkáppen kivesz két élet a 
körútból, és újra összeköti a két 
keletkezett útat. Ezt 2-opt lépésként szokták emlegetni. 
Csak egyetlen módon lehet a két keletkezett útat 
összekötni úgy, hogy körútat kapjunk. Ezt csak akkor 
tesszük meg, ha a keletkezendõ körút rövidebb lesz. Addig 
veszünk ki éleket, és kötjük össze a keletkezett 
utakat, ameddig már nem lehet javítani az úton. Igy a 
körút 2-optimális lesz. A következõ kép egy gráfot 
mutat amelyen végrehajtunk egy 2-opt mozdulatot.

\subsubsection{A genetikus algoritmus}

A gentikus algoritmusok biologia alapú algoritmusok. Az 
evolúciót utánozzák, és ez által fejlesztenek ki 
megoldásokat, sokszor nagyon nehéz feladatokra. Az általános 
gentikus algoritmusnak a következõ lépéseit 
különböztethetjük meg:

\emph{1. Létrehoz egy véletlenszerû kezdeti állapotot}

Egy kezdeti populációt hozunk létre, véletlenszerûen a 
lehetséges megoldásokból. Ezeket kromoszómáknak 
nevezzük. Ez különbözik a szimbolikus MI rendszerektõl, ahol 
a kezdeti állapot egy feladatra nézve adott.

\emph{2. Kiszámítja a megoldások alkalmasságát}

Minden kromoszómához egy alkalmasságot rendelünk, attól 
függõen, hogy mennyire jó megoldást nyújt a feladatra. 

\emph{3. Keresztezés}

Az alkalmasságok alapján kivalásztunk néhány 
kromoszómát, és ezeket keresztezzük. Eredményûl két 
kromoszómát kapunk, amelyek az apa és az anya kromoszóma 
génjeinek a kombinációjából állnak. Ezt a folyamatot 
keresztezésnek nevezzük. Keresztezéskor tulajdonképpen 
két részmegoldást vonunk össze, és azt reméljük, 
hogy egy jobb megoldás fog keletkezni.

\emph{4. A következõ generáció létrehozása}

Ha valamelyik a kromoszómák közül tartalmaz egy megoldást, 
amely elég közel van, vagy egyenlõ a keresett megoldással, akkor 
azt mondhatjuk , hogy megtaláltuk a megoldást a feladatra. Ha ez a 
feltétel nem teljesûl, akkor a következõ generáció is át 
fog menni az \textbf{a.-c}. lépéseken. Ez addig folytatódik, 
ameddig egy megoldást találunk.
