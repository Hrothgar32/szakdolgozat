% BEÁLLÍTÁSOK - JOBB NEM VÁLTOZTATNI
\documentclass[final,12pt]{ubb_dolgozat}
\usepackage{definitions}


% milyen nyelveken akarunk forráskódot megjeleníteni
\lstloadlanguages{Java,Python}
% más lehetőségek:
% C, Matlab, Mathematica, Octave, Pascal, Perl, Python
% SCilab, SQL, Haskell, Lisp, Lua, make, ML, PHP, Prolog
%
% a teljes lista a LISTINGS csomagban.


% ezt be lehet tenni MINDEGYIK megjelenítendő kód elé opcióként
\lstset{language=Java}


%%%%%%%%%%%%%%%%%%%%%%%%%%%%%%%%%%%%%%%%%%%%%%%
%%!!          EZT KELL VÁLTOZTATNI       !!%%%%
%%     A DOLGOZAT CÍMOLDALÁNAK ELEMEI        %%

% Az alábbi sorokat ki kell tölteni!!

% mikor védünk
\submityear{2022}

% dolgozattípus
\doctypeHU{Szakdolgozat}
% \doctypeHU{Magiszteri dolgozat}
\doctypeEN{Diploma Thesis}
% \doctypeEN{Master's Thesis}
\doctypeRO{Lucrare de licență}
% \doctypeRO{Lucrare de disertație}

% szak, tanulmányi program
\specHU{Informatika}
% \specHU{Vállalati Szoftvertervezés és Fejlesztés}
\specEN{Computer Science}
% \specEN{Enterprise Software Design and Development}
\specRO{Informatică}
% \specRO{Proiectarea \c{s}i Dezvoltarea Aplica\c{t}iilor Enterprise}

% cím
\titleHU{Szakdolgozat cím}
\titleEN{License thesis title}
\titleRO{Titlu lucrare licență}

% szerző
\authorHU{Gáspár Adalbert}
\authorEN{Adalbert Gáspár}
\authorRO{Adalbert Gáspár}

% témavezető
\tutorHU{dr. Oktató Ottó,\\egyetemi adjunktus}
\tutorEN{Ottó Oktató, PhD.\\University lecturer}
\tutorRO{Lector dr. Ottó Oktató}
% a hozzátartozás akkor szükség, ha NEM BBTE-s a tanár
%{\large Babe\c{s}--Bolyai Tudományegyetem,\\
% Matematika és Informatika Kar}% ha különbözik, akkor fel kell tûntetni

 
%\includeonly{bevezet}


\begin{document}

% nincs oldalszámozás az elején
\pagenumbering{gobble}

% címoldalak
\maketitle

%% angol absztrakt
\begin{abstractEN}

% a lenti részt értelemszerûen ki kell tölteni a dolgozat angol kivonatával.
% A BEGIN ... END között CSAK A SAJÁT SZÖVEG kell, hogy legyen.
% Az utolsó mondatot benne kell hagyni, mely által kijelentitek, hogy a munkátok SAJÁT.

{

	\vfill
  
  \center{
  
  {\huge EZ AZ OLDAL NEM RÉSZE A DOLGOZATNAK!}
	
	\vspace{0.5cm}
	
	{\Large Ez az angol kivonatot.
 
        Tükrözze a dolgozat tartalmát és eredményeit.}
	
	\vspace{0.5cm}
	
	{\huge A DOLGOZATTAL EGYÜTT KELL BEADNI!}
  
  }
	
	\vfill

	Kötelező befejezés:
}
\vspace*{.5cm}

This work is the result of my own activity. I have neither given nor received unauthorized assistance on this work.

\end{abstractEN}

%% 

% a dolgozat tartalomjegyzéke -- ez automatikusan generálódik a STRUKTÚRA alapján.
{ \baselineskip 3ex
  \parskip 1ex
  \tableofcontents
  \clearpage
}


% számozás kezdődik innen
\pagenumbering{arabic}
\setcounter{page}{1}

%%%%%%%%%%%%%%%%%%%%%%%%%%%%%%%%%%%%%%%%%%%%%%%%%%%%%%%%%%%
%%%%%%%%%%         a dolgozat tartalma         %%%%%%%%%%%%

% ajánlott külön file-okba írni az egyes fejezeteket,
% ugyanis úgy jobban át lehet látni.


% a bevezetõ fejezet FILE-ja.
\include{bevezet}

%!TEX root = minta_dolgozat.tex
%%%%%%%%%%%%%%%%%%%%%%%%%%%%%%%%%%%%%%%%%%%%%%%%%%%%%%%%%%%%%%%%%%%%%%%
\chapter{Alapok}\label{ch:ALAP}
%%%%%%%%%%%%%%%%%%%%%%%%%%%%%%%%%%%%%%%%%%%%%%%%%%%%%%%%%%%%%%%%%%%%%%%

\begin{osszefoglal}
	A fejezetek elején egy rövid összefoglalót teszünk. Ez a rész opcionális.
	
\end{osszefoglal}

%%%%%%%%%%%%%%%%%%%%%%%%%%%%%%%%%%%%%%%%%%%%%%%%%%%%%%%%%%%%%%%%%%%%%%%
\section{A gépi tanulás}\label{sec:ALAP:ml}

A gépi tanulás neve  \citeN{Mitchell97} azonos címû -- ``{\em Machine Learning}'' -- könyvéből származtatható.
A könyv alapján azt a kutatási területet nevezzük így, amelyben a cél olyan programok írása, amelyek  futtatásuk során fejlődnek, vagyis valamilyen szempont szerint jobbak, okosabbak lesznek.%
\footnote{ %
Részlet \citeN{Mitchell97} könyvéből:\newline
    ``The field of machine learning is concerned with the question of how to construct computer programs    that automatically improve with experience.''
}  %
Itt az ``okosság'' metaforikus: a futási idő folyamán valamilyen mérhető jellemzőnek a javulását értjük alatta.
%
Például a felhasználás kezdetén a szövegfelismerő még nem képes a szövegek azonosítására, azonban a használat -- és a felhasználói utasítások -- után úgy módosítja a működési paramétereit, hogy a karakterek egyre nagyobb hányadát tudja felismerni.


%%%%%%%%%%%%%%%%%%%%%%%%%%%%%%%%%%%%%%%%%%%%%%%%%%%%%%%%%%%%%%%%%%%%%%%
\section{Adatelemzés}\label{sec:ALAP:adatelem}

A mesterséges intelligencia azon módszereit, amelyeket numerikus
vagy {\em
  enyhén strukturált}%
\footnote{%
 {\bf Enyhén} strukturált (nagyon felületesen): az adatok komponensei
  (dimenziók) közötti kapcsolat nem túl bonyolult.
}%
adatokra tudunk alkalmazni, gépi tanulásos módszereknek nevezzük
\cite{Mitchell97}.  A gépi tanulás e meghatározás alapján egy
szerteágazó tudományág, amelynek keretén belül sok módszerről és
ennek megfelelően sok alkalmazási területről beszélhetünk. A
korábban említett neurális modellekkel ellentétben a központban
itt az adatok vannak: azok típusától függően választunk például a
binomiális modell és a normális eloszlás, a fő- vagy
független-komponensek módszere vagy a k-közép és EM
algoritmusok között.
Egyre több adatunk van, azonban az ``információt'' megtalálni egyre nehezebb.%
\footnote{%
    David Donoho, a Stanford Egyetem professzora szerint a XXI. századot az adatok határozzák meg: azok gyûjtése, szállítása, tárolása, megjelenítése, illetve az adatok {\bf felhasználása}.
}%


%%%%%%%%%%%%%%%%%%%%%%%%%%%%%%%%%%%%%%%%%%%%%%%%%%%%%%%%%%%%%%%%%%%%%%%
\section{Szerkesztés}\label{sec:ALAP:szerkeszt}

A következőkben áttekintjük a \LaTeX dokumentumok szerkesztésének alapjait.

Átfogó referenciák a következők:
\begin{description}%
	\item[\cite{LatexNotSoShort}] -- egy \LaTeX gyorstalpaló. A könyvben nagyon
	  célirányosan mutatják be a szerkesztési szabályokat és a fontosabb parancsokat.
	\item[\cite{LatexNotSoShortHU}] -- a gyorstalpaló magyar változata egy BME-s
	  csapat jóvoltából.
	\item[\cite{Doob95,MittelbachEtAl04}] -- az angol nyelvû alapkönyvek. Kérésre az
	  utóbbi -- \cite{MittelbachEtAl04} -- elérhetővé tehető.
\end{description}
Természetesen a Kari könyvtárban is találtok könyvészetet. A fentebb említetteken kívül nagyon sok internetes oldal tartalmaz \LaTeX szerkesztésről bemutatókat:


\begin{figure}[t]
  \centering
  \pgfimage[width=0.2\linewidth]{images/bayes}
  \caption[Példa képek beszúrására]%
  {Példa képek beszúrására: a képen rev. Thomas Bayes látható. A képek után {\em kötelezően} szerepelnie kell a forrásnak:~\url{http://en.wikipedia.org/wiki/Thomas_Bayes}}
  \label{fig:ALAP:sm1}
\end{figure}

Képeket beszúrni a\\
 \verb+\pgfimage[width=0.4\linewidth]{images/bayes}+\\
paranccsal lehet, ahol a\\
\verb+width=0.4\linewidth+ \\
a kép szélességét jelenti. Amennyiben a magasság nincs megadva -- mintjelen esetben -- akkor azt automatikusan számítja ki a rendszer, az eredeti kép arányait figyelembe véve.
Egy másik jellegzetesség az, hogy a képek kiterjesztését nem adjuk meg -- a \LaTeX megkeresi a számára elfogadható kiterjesztéseket, azok listájából az elsőt használja. A .JPG, .PNG, .TIFF, valamint a .PDF kiterjesztések is használhatók.

Két kép egymás mellé tétele a \verb+tabular+ környezet-mintával lehetséges, amint a \ref{fig:ALAP:sm2} ábrán látjuk.
Amennyiben grafikonunk van, általában ajánlott VEKTOROSAN menteni -- ezt általában a PDF driverrel tesszük -- az eredmény a \ref{fig:ALAP:sm3} ábrán látható.

\begin{figure}[t]
  \centering
  \begin{tabular}{ccc}
		  \pgfimage[height=4cm]{images/bayes}
		  &
		  \pgfimage[height=4cm]{images/vapnik}
	\end{tabular}
  \caption[Példa képek beszúrására egy táblában]%
  {Példa képek beszúrására: a bal oldalon rev. Thomas Bayes, a jobb oldalon egy jelenkori matematikus, Vladimir Vapnik látható.\\
  {\white .}\hfill\url{http://en.wikipedia.org/wiki/Vladimir_Vapnik}}
  \label{fig:ALAP:sm2}
\end{figure}

\begin{figure}[t]
  \centering
  \pgfimage[width=0.7\linewidth]{images/parzen}
  \caption[Példa grafika beszúrására]%
  {Példa grafika beszúrására.}
  \label{fig:ALAP:sm3}
\end{figure}

A szerkesztés folyamata során ajánlott a:
\begin{itemize}
	\item különböző strukturális elemek használata: a \verb+\chapter+, \verb+\section+, \verb+\subsection+, \verb+\subsubsection+ parancsok.
	\item a listák használata felsorolásoknál;
  \item a tétel típusú environmentek és a bizonyítás-environment használata (alább bemutatunk néhányat, az összes értelmezett tétel típust megtalálod a definitions.sty fájlban).
\end{itemize}

\begin{ert}
  Az $n$ pozitív egész számot \emph{négyzetmentesnek} nevezzük, ha az $n$ prímtényezős felbontásában minden prím legfentebb az első hatványon szerepel.
\end{ert}
\begin{pld}
  Az 1, 7 és 33 természetes számok négyzetmentesek, míg a 9 és a 45 nem.
\end{pld}
\begin{tet}\label{tet:negyzment}
  Az $n$ pozitív egész szám akkor és csak akkor négyzetmentes, ha minden $n$ elemű Abel csoport ciklikus.
\end{tet}
\begin{proof}
  Túl hosszú!
\end{proof}
\begin{meg}
  \Aref{tet:negyzment} tétel a végesen generált Abel csoportok jellemzési tételének a következménye. 
\end{meg}

%!TEX root = dolgozat.tex
%%%%%%%%%%%%%%%%%%%%%%%%%%%%%%%%%%%%%%%%%%%%%%%%%%%%%%%%%%%%%%%%%%%%%%%
\chapter{Diszkrimináns-módszerek}\label{ch:diszkr}

\begin{osszefoglal}
	Példafejezet. Nem releváns a szöveg.
	
	A fejezetben a matematikai elemeket illusztráltuk.
\end{osszefoglal}



\section{Lineáris diszkrimináns}

Legyen ismert az $x_1, \ldots , x_n$ gyakorló minták sorozata és a minták 
osztályozása. A minták száma $N$, az $\Omega_x$ mintatér $d$ dimenziója 
sokkal kisebb, mint $N$.

Célunk egy olyan függvény meghatározása, mely \emph{diszkriminál} az adatok terében, azaz a pozitív példákra pozitív értékkel, a negatívakra pedig negatív értékkel tér vissza:
\begin{equation}
	f: \Omega_x \rightarrow {\mathbb{R}}
	\quad\text{ úgy, hogy }
	\quad f(x)
	\begin{cases}
		<0 & \forall x\in Neg \\
		\geq 0 & \forall x\in Poz
	\end{cases}
	\label{eq:diszkr:fugg}
\end{equation}
ahol a negatív doméniumot $Neg$-gel, a pozitívet meg $Poz$-zal jelöltük.%

{\footnotesize
	Figyeljük meg a \LaTeX kódban a \verb!\label! használatát: a szövegben kijelentünk egy \emph{címkét}, melyet az \protect{\verb+\eqref{eq:diszkr:fugg}+} vagy a \protect{\verb+\ref{eq:diszkr:fugg}+} parancsokkal tudunk késõbb beszúrni a szövegbe.
  A kompilálás során a \LaTeX mindig aktualizálja a mutató értékét, nem kell tehát újraszámozni kézzel az objektumokat. Ez természetesen érvényes más számozott egységre is, mint fejezetek, alfejezetek, bekezdések, ábrák, stb. -- lásd a jelen fejezet TEX kódját.
}

{\bf Lineáris diszkrimináns függvény} 

A diszkrimináns függvények 
legegyszerûbb változata a lineáris. A lineáris diszkrimináns függvényeket a 
következõképp definiáljuk: 

$$D_k(x)=x_1\alpha{1k}+\ldots 
+x_N\alpha_{Nk}+\alpha_{N+1,k}\;\;\;k=1,2,\ldots ,K,$$ ahol $K$ az osztályok száma, $x_1, \ldots, x_N$ az $x$ 
mintavektor $N$ komponense, az $\alpha$ számok a súlyozó együtthatók. Vektoros 
formában felírva:

$$D_k(x)=\tilde{x}^T\alpha_k=\alpha_k^T\tilde{x},$$

ahol $\tilde{x}^T=[x^T,1]$ a transzponáltja $\tilde{x}$-nak, a 
megnövelt mintavektornak, és $\alpha_k$ a $k$-adik súlyozó vektor, amely 
tartalmazza az $N+1$ súlyozó együtthatót.

%!TEX root = dolgozat.tex
%%%%%%%%%%%%%%%%%%%%%%%%%%%%%%%%%%%%%%%%%%%%%%%%%%%%%%%%%%%%%%%%%%%%%%%
\chapter{Matlab és Netlab ismertetõ}\label{ch:MAT}

\begin{osszefoglal}
	A következõkben a jegyzet során használt Matlab nyelvet
	mutatjuk be és definiáljuk a használt függvényeket. A Matlab programnyelven írták meg a NETLAB csomagot, mellyel nagyon könnyen lehet mintafelismerõ algoritmusokat elemezni.
\end{osszefoglal}


%%%%%%%%%%%%%%%%%%%%%%%%%%%%%%%%%%%%%%%%%%%%%%%%%%%%%%%%%%%%%%%%%%%%%%%
\section{Rövid ismertetõ}\label{sec:MAT:bev}

A Matlab nyelv egy {\em interpreter}. A változókat létrehozzuk,
nincs szükség azok deklarálására. A változókat
megfeleltetésekkel  hozzuk létre. A változók lehetnek: {
% \renewcommand{\baselinestretch}{0.9}\normalsize
\begin{description}
    \setlength{\itemsep}{0.04mm}
    \item[valós] típusúak -- például \code{a=ones(5,1);}, amelyekrõl a rendszer megjegyzi, hogy mekkorák és a megfelelõ mennyiségû memóriát lefoglalja. A változók alapértelmezetten mátrixok, azonban lehetõség van magasabb fokú tenzorok definíciójára is, például az \code{b=ones(5,5,2);} egy $5\times 5\times 2$-es méretû tenzort hoz létre, mely két darab $5\times 5$-ös mátrixot tárol és melyekre a \code{b(:,:,1)},  valamint a \code{b(:,:,2)} parancsokkal hivatkozunk. Egy mátrixban egy egész sort {\em kettõspont}tal választunk ki. A Matlab-ban nincsenek egész vagy logikai típusú változók.
    \item[sztring] típusúak -- például \code{s='a1b2c3d4'} egy sztringet hoz létre. A sztringek karakter típusú vektorok, melyekkel az összes mátrix mûvelet is végezhetõ.
    \item[cella] típusúak -- például \code{c=\{'sty',[1;2;3;4;5;6],2\}}, mindegyik elem lehet különbözõ típusú és méretû. A cellák elemeire a \code{c\{3\}} jelöléssel hivatkozunk és nem tudunk a vektorokra illetve a mátrixokra jellemzõ mûveletek alkalmazni azokon.
\end{description}
} Mint általában, a Matlab rendszerben is segít a \code{help}
parancs, mely egy adott parancshoz ad magyarázatot: a \code{help
<függvény>} a függvényhez tartozó magyarázatot jeleníti meg. A
rendszerbe be van építve egy további segítség, a \code{demo}
parancs, mely példákon keresztül mutatja be a {Matlab} mûködését
és az interpreter jelleg által nyújtott lehetõségeket.

%%%%%%%%%%%%%%%%%%%%%%%%%%%%%%%%%%%%%%%%%%%%%%%%%%%%%%%%%%%%%%%%%%%%%%%
\section{Matlab mûveletek}\label{sec:MAT:muv}

A {Matlab} nyelvben a vektorokra jellemzõ mûveletek jelölése
intuitív: a mátrix transzponáltját a \code{bt=b'} mûvelet, a
mátrix-szorzatot a \code{c=b(:,:,1)*a} jelöli. Amennyiben a
mûveletek operandusai nem megfelelõ méretûek, a rendszer
hibaüzenetet ír ki. A szokásos aritmetikai mûveleteken kívül
ismeri a rendszer a hatványozást is, a $\hat{~}$ jelöléssel,
melyek mind érvényesek a mátrixokra is. Az osztáshoz például két
mûvelet is tartozik, melyek az $A\cdot X = B\; \Leftrightarrow \;
X=A\backslash B$, valamint a $X\cdot A = B \; \Leftrightarrow \;
X=A/B$ lineáris egyenleteket oldják meg.

Sokszor szeretnénk, ha elemenként végezne a rendszer mûveleteket a mátrixokon, ezt a mûveleteknek pontokkal való prepozíciójával tesszük. Például a \code{C=(b(:,:,2))$\hat{~}$2} a mátrix önmagával való szorzásának az eredményét, a \code{C=(b(:,:,2)).$\hat{~}$2} a mátrix elemeinek a négyzeteit tartalmazó mátrixot adja eredményül.

%%%%%%%%%%%%%%%%%%%%%%%%%%%%%%%%%%%%%%%%%%%%%%%%%%%%%%%%%%%%%%%%%%%%%%%
\section{Matlab függvények}\label{sec:MAT:fugg}

A jegyzet során a programokban gyakran alkalmaztuk a következõ függvényeket:
% { \renewcommand{\baselinestretch}{0.98}\normalsize
\begin{description}
    \setlength{\itemsep}{0.04mm}
    \item[rand] -- egy véletlen számot térít vissza a $[0,1]$ intervallumból az intervallumon egyenletes eloszlást feltételezve. Argumentum nélkül a függvény egy nulla és egy közötti véletlen számot, egy argumentummal egy $k\times k$ méterû véletlen mátrixot, egy vektorra pedig egy tenzort térít vissza, melynek méreteit a vektor elemei tartalmazzák. A {\bf randn} hasonlóan véletlen változókat visszatérítõ függvény, azonban azok nulla átlagú és egy szórású normális eloszlást követnek.
    \item[ones] -- a fenti esethez hasonlóan egy vektort, mátrixot vagy tenzort térít vissza, azonban az elemeket $1$-gyel tölti fel. A {\bf zeros} az elemeket lenullázza.
    \item[linspace] -- a bemenõ argumentumok skalárisak és a függvény visszatéríti az elsõ két argumentum mint intervallum $N$ részre való felosztásának a vektorát; az $N$ a harmadik bemenõ változó.
    \item[randperm] -- egy argumentummal a $k$ hosszúságú $[1,\ldots,k]$ vektor egy véletlen permutációját téríti vissza.
    \item[union] -- a lista elemeit halmazként használva egyesíti a két bemenõ halmazt.
    \item[setdiff] -- a lista elemeit halmazként használva visszatéríti az argumentumok metszetét. Használhatjuk még a {\bf unique} és az {\bf ismember} parancsokat a halmazzal való mûveletekre.
    \item[find] -- indexeket térít vissza. Egy vektor azon elemeinek indexét, mely egy bizonyos feltételnek eleget tesz. Használják a vektor elemeinek szelekciójára.
    \item[repmat] -- elsõ argumentuma egy mátrix, amit adott sokszorossággal bemásol az eredménybe. A sokszorosságot a második argumentum adja meg.
    \item[reshape] -- argumentumai egy vektor vagy mátrix illetve egy méret paraméter. A függvény az elsõ argumentum elemeit a második argumentumban található méretek szerint formázza át. Pl. a \code{v} $256$ hosszú vektort az \code{m} $16\times 16$-os vektorba a \code{m=reshape(v,[16,16])} paranccsal alakítjuk át.
    \item[meshgrid] -- az elsõ argumentum elemeit $X$-tengelyként, a második argumentumét $Y$-tengelyként használva két $m\times n$-es mátrixot térít vissza, ahol az elemek a négyzetháló $X$, illetve $Y$ koordinátái oszlop-szerinti bejárásban. Hasznos amikor egy felületet szeretnénk kirajzolni.
    \item[diag] -- ha a bemenõ argumentum egy mátrix, akkor az átlón levõ elemek vektora az eredmény, ha pedig egy vektor, akkor az az átlós mátrix, mely nulla a fõátló elemeit kivéve, ott meg a bemenõ vektor elemei találhatók. Igaz a \code{b = diag(diag(b))} állítás.
    \item[inline] -- egy függvény, mely segít rövid függvényeket -- általában egysorosakat -- definiálni, a függvényekben az alapértelmezett argumentum az $x$, több argumentum esetén a sorrendet is lehet specifikálni.
    \item[load] -- egy korábban elmentett állapottér változóit állítja vissza. A változók elmentéséhez használjuk a {\bf save} parancsot.
    \item[fprintf] -- egy vektor eleit formázva kiírja, a 'C'-hez hasonló szintakszisban. Hasonlóan mûködnek a {\bf disp}, a {\bf sprintf}, valamint a {\bf num2str} parancsok.
    \item[acosd] -- az argumentumra alkalmazott inverz koszinusz függvény, fokokban kifejezve. Más trigonometriai függvények a szokásosak: {\bf sin}, {\bf cos}, {\bf tan}, {\bf atan}, melyeket lehet elemenként és egyben is alkalmazni, ekkor az eredmény a vektorok elemeire alkalmazott mûveletek vektora.
    \item[chol] -- egy négyzetes pozitív definit mátrix Cholesky-felbontása. Az $A$ mátrix Cholesky-alakja egy olyan $C$ mátrix, mely csak a fõátlón és az alatt tartalmaz nullától különbözõ elemeket {\em és} fennáll az $A = C\cdot C^T$ egyenlõség. Figyeljük meg, hogy a Cholesky-alakot a négyzetgyök általánosításának lehet tekinteni.
    \item[plot] -- két vektort megadva kirajzolja az $(x_1(i),x_2(i))$ pontokat és azokat összeköti egy vonallal. Egy argumentum esetén $x_1=[1,\ldots,N]$ és $x_2$ a bemenõ paraméter. Opcionálisan lehet stílusparamétereket is megadni. A {\bf plot3} paranccsal háromdimenzióban lehet pontokat, vonalakat megjeleníteni.
    \item[hist] -- egy adott adatvektor elemeinek a gyakoriságát rajzolja ki. Alapértelmezetten a vektor legkisebb és legnagyobb eleme közötti intervallumot osztja fel $10$ részre és számolja az egyes szakaszokba esõ pontok számát. Úgy az intervallum mérete, mint a részintervallumok számossága illetve mérete változhat.
    \item[contour] -- egy $Z$ mátrix által definiált felület kontúrjait rajzolja ki. A felület generálásánál általában használjuk a {\bf meshgrid} parancsot. A felületek rajzolására használhatjuk a {\bf surf} parancsot is.
    \item[figure] -- létrehoz egy új ábrát illetve, amennyiben létezik a kívánt ábra, akkor aktívvá teszi azt.
    \item[subfigure mnk] -- létrehoz az ábrán egy részábrát úgy, hogy az eredeti ábra terét $m\times n$ részre osztja, majd annak a $k$-adik komponensét teszi aktívvá.
    \item[xlim] -- az aktuális rajzon beállítja az $X$ tengely alsó és felsõ határát. Ugyanígy mûködnek az {\bf ylim} és {\bf zlim} parancsok az $Y$ illetve a $Z$ tengelyekre.
    \item[quadprog] -- az $x^THx + b^Tx + c$ másodfokú egyenlet minimumát határozza meg, ahol feltételezzük, hogy a megoldásokat az $A x>0$ konvex doméniumra szûkítjük. Bõvebb információk \citeN{Boyd04} könyvében.
\end{description}
% }

%%%%%%%%%%%%%%%%%%%%%%%%%%%%%%%%%%%%%%%%%%%%%%%%%%%%%%%%%%%%%%%%%%%%%%%
\section{Netlab bevezetõ}\label{sec:MAT:Netlab}

A {Netlab} neurális modellek hatékony implementációit
tartalmazza. A programcsomag egy egséges felületet nyújt a létezõ
algoritmusok gyors teszteléséhez és az új algoritmusok írásához.
A különbözõ módszerek közös jellemzõje, hogy egy változóba --
általában ennek neven \code{net} és egy struktúra -- gyûjti össze
a modell paramétereit. Ehhez általában elõször specifikáljuk a
modellt; ennek a függvénynek a neve ugyanaz, mint a modell neve.
Amennyiben szükséges, akkor a \code{<modellnév>init} paranccsal
lehet más paramétereket is beállítani. A modelleket a
\code{<modellnév>train} paranccsal lehet tanítani, ahol általában
paraméterként kerül a tanuló adathalmaz illetve az optimalizálási
folyamatot jellemzõ más konstansok. Amikor megvan az eredmény,
akkor a tanult -- becsült -- modell paramétereit használjuk a
\code{<modellnév>fwd} paranccsal. Ahol nem lehetséges az új
adatokra a tesztelés, ott a paraméterek terében tudunk mintát
vételezni a \code{<modellnév>sample} függvény segítségével.

Az általunk használt modellek a következõk:
\begin{description}
    \setlength{\itemsep}{0.04mm}
    \item[mlp] -- a többrétegû neurális háló;
    \item[rbf] -- az RBF típusú háló;
    \item[kmeans] -- a k-közép algoritmus;
    \item[som] -- a SOM vagy Kohonen-háló;
\end{description}

A \code{net} struktúrának van egy azonosítója, a \code{net.type}
mezõ és a többi paraméter ennek az azonosítónak is a függvénye.
További mezõk a bemenõ illetve kimeneti adatok dimenzióit, az
aktivációs függvények típusait, valamint a különbözõ
kapcsolatokhoz rendelt súlymátrixokat tárolják.

Az optimalizálás szintén egységesen történik, minden modellnek van
hibafüggvénye, ezt a \code{<modellnév>err} függvény tartalmazza.
Az optimalizálási rutin a \code{netopt}, mely a struktúrát, az
adatokat, valamint egy \code{options} vektort kap paraméterként és
a visszaadott struktúra tartalmazza az optimalizált modellt.
Ahhoz, hogy heterogén struktúrájú modelleket lehessen használni,
minden modellhez kell írjunk egy \code{<modellnév>pak}, illetve
egy \code{<modellnév>unpak} függvényt, mely a paramétereket a
struktúrából egy vektorba, illetve visszaalakítja. Az
\code{options} vektor a \code{netopt} függvényt paraméterezi. Egy
$14$ hosszúságú vektor, melynek fõbb értékei: {
\renewcommand{\baselinestretch}{0.98}\normalsize
\begin{description}
    \setlength{\itemsep}{0.04mm}
    \item[\code{options(1)}] -- a hibafüggvény értékeinek a kiírása. $+1$-re minden lépésben kiírja a hibát, nullára csak a végén, negatív értéknél nem jelenít meg semmit;
    \item[\code{options(2)}] -- a megállási feltétel abszolút pontossága: amennyiben két egymásutáni lépésben az $\theta$ paraméterek kevesebbet változnak, akkor az algoritmus leáll;
    \item[\code{options(3)}] -- az \code{options(2)}-höz hasonló küszöbérték, azonban ez a hibafüggvény értékeit vizsgálja;

    \item[\code{options(10)}] -- tárolja és visszaadja a hibafüggvény kiértékelésének a számát;

    \item[\code{options(11)}] -- tárolja és visszaadja a hibafüggvény gradiense hívásának a számát;

    \item[\code{options(14)}] -- a lépések maximális száma, alapértelmezetten $100$.
\end{description}
}

A {Netlab} csomagban implementálva van sok hasznos lineáris és
nemlineáris modell, mint például a PCA módszer, valamint annak valószínûségi
kiterjesztése, a {\bf ppca} módszer. Megtalálható az
általánosított lineáris modell -- {\bf glm} --, számos konjugált
gradiens módszer és sok más. Jelen felsorolásban említettünk
néhányat a használt illetve a további feladatok során használható
programok közül, ezt a teljesség igénye nélkül tettük, az
érdeklõdõ hallgatónak ajánljuk a {Netlab} hivatalos honlapját a
\url{http://www.ncrg.aston.ac.uk/netlab}\label{link:netlab2}
oldalt és \citeN{Nabney02} {Netlab} könyvét.


%!TEX root = dolgozat.tex
%%%%%%%%%%%%%%%%%%%%%%%%%%%%%%%%%%%%%%%%%%%%%%%%%%%%%%%%%%%%%%%%%%%%%%%
\chapter{Eredmények bemutatása és értékelése}\label{ch:elemzes}


\section{Az utazóügynök feladata}

A standard utazóügynök feladat a következõ:

Adott egy súlyozott gráf G = (V,E) a cij súly az i és j csomópontokat
összekötõ élre vontakozik, és a cij érteke egy pozitiv szám. Találd meg azt a
körutat, amelynek minimális a költsége.

\section{Az utazóügynök feladatára vonatkozó heurisztikák}

Az utazóügynök feladata egy np-teljes feladat. A megoldás megtalálására túl
sok idõ szükséges, ezért heurisztikákat használunk.

\subsection{Beszúrási herusztika}

A beszúrási heurisztikát akkor használjuk, amikor egy új 
csomópontot akarunk beszúrni a körútba. úgy szúrunk 
be egy új csomópontot , hogy a körút hossza minimális 
legyen. A csomópontot minden pozícióba megpróbáljuk 
beszúrni, és mindig kiszámoljuk a költséget. Az új 
pozícioja a csomópontnak a körútban az a pozício lesz, 
ahol a költség minimális.

\subsection{Körútjavító heurisztika}

A körút javito heurisztikakat arra használjuk, hogy meglévõ 
megoldásokat javitsunk. A legismertebb heurisztikak a 2-opt és a 
3-opt heurisztikak.

\subsubsection{A 2-opt herurisztika}

A 2 opt heurisztika megprobal talalni 2 elet, amelyet el lehet tavolitani, 
és 2 elet, amelyet be lehet szúri, ugy, hogy egy körútat 
kapjunk, amelynek a költsége kisebb mint az eredetié. Euklideszi 
távolságoknál a 2-opt csere, kiegyenesíti a 
körútat, amely saját magát keresztezi.

A 2-opt algoritmus tulajdonkáppen kivesz két élet a 
körútból, és újra összeköti a két 
keletkezett útat. Ezt 2-opt lépésként szokták emlegetni. 
Csak egyetlen módon lehet a két keletkezett útat 
összekötni úgy, hogy körútat kapjunk. Ezt csak akkor 
tesszük meg, ha a keletkezendõ körút rövidebb lesz. Addig 
veszünk ki éleket, és kötjük össze a keletkezett 
utakat, ameddig már nem lehet javítani az úton. Igy a 
körút 2-optimális lesz. A következõ kép egy gráfot 
mutat amelyen végrehajtunk egy 2-opt mozdulatot.

\subsubsection{A genetikus algoritmus}

A gentikus algoritmusok biologia alapú algoritmusok. Az 
evolúciót utánozzák, és ez által fejlesztenek ki 
megoldásokat, sokszor nagyon nehéz feladatokra. Az általános 
gentikus algoritmusnak a következõ lépéseit 
különböztethetjük meg:

\emph{1. Létrehoz egy véletlenszerû kezdeti állapotot}

Egy kezdeti populációt hozunk létre, véletlenszerûen a 
lehetséges megoldásokból. Ezeket kromoszómáknak 
nevezzük. Ez különbözik a szimbolikus MI rendszerektõl, ahol 
a kezdeti állapot egy feladatra nézve adott.

\emph{2. Kiszámítja a megoldások alkalmasságát}

Minden kromoszómához egy alkalmasságot rendelünk, attól 
függõen, hogy mennyire jó megoldást nyújt a feladatra. 

\emph{3. Keresztezés}

Az alkalmasságok alapján kivalásztunk néhány 
kromoszómát, és ezeket keresztezzük. Eredményûl két 
kromoszómát kapunk, amelyek az apa és az anya kromoszóma 
génjeinek a kombinációjából állnak. Ezt a folyamatot 
keresztezésnek nevezzük. Keresztezéskor tulajdonképpen 
két részmegoldást vonunk össze, és azt reméljük, 
hogy egy jobb megoldás fog keletkezni.

\emph{4. A következõ generáció létrehozása}

Ha valamelyik a kromoszómák közül tartalmaz egy megoldást, 
amely elég közel van, vagy egyenlõ a keresett megoldással, akkor 
azt mondhatjuk , hogy megtaláltuk a megoldást a feladatra. Ha ez a 
feltétel nem teljesûl, akkor a következõ generáció is át 
fog menni az \textbf{a.-c}. lépéseken. Ez addig folytatódik, 
ameddig egy megoldást találunk.



\appendix
%!TEX root = dolgozat.tex
%%%%%%%%%%%%%%%%%%%%%%%%%%%%%%%%%%%%%%%%%%%%%%%%%%%%%%%%%%%%
\chapter{Fontosabb programkódok listája}\label{ch:progik}

%%%%%%%%%%%%%%%%%%%%%%%%%%%%%%%%%%%%%%%%%%%%%%%%%%%%%%%%%%%%%%%%%

Itt van valamennyi Prolog kód, megfelelõen magyarázva (komment-elve). A programok beszúrása az\\
\verb+\lstinputlisting[multicols=2]{progfiles/lolepes.pl}+\\
paranccsal történik, és látjuk, hogy a példában a \verb+progfiles+ könyvtárba tettük a file-okat.

Az alábbi kód Prolog nyelvbõl példa. Az \verb+\lstset{language=Prolog}+ paranccsal a programnyelvet változtathatjuk meg, ezt a \code{listings} csomag teszi lehetõvé \cite{listingCite}, amely nagyon jól dokumentált.

\lstinputlisting[language=Prolog]{progfiles/lolepes.pl}
%
% opciók:
% ,multicols=2

\section{Kódrészletek beszúrása \texttt{minted}del}

\begin{listing}[b!]
  \inputminted{java}{progfiles/HelloWorld.java}
  \caption{Példakódrészlet Java-ban}
  \label{lst:javahw}
\end{listing}

Egy szintaxiskiemelt Java példaalkalmazást láthatunk \aref{lst:javahw}. kódrészletben. Ezek a részletek is ábrának számítanak, ezért számozva vannak, utalunk rájuk s ráhagyjuk a \LaTeX-re az elrendezésüket. Ha csak egy rövid részletet szeretnénk beszúrni, ezt megtehetjük inline módon is a következőképpen:

\begin{minted}{java}
    System.out.println("I am some inline code");
\end{minted}



{ \renewcommand{\baselinestretch}{0.8}
  \normalsize 
  \setlength{\itemsep}{-2.4mm}
  \setlength{\bibspacing}{0.67\baselineskip}
  \bibliographystyle{abbrvnat_hu}
  \bibliography{dolgozat}
}

\end{document}
